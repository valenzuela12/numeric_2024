\documentclass[12pt]{article}
  
\include{latex_env}

\rhead{Worksheet Three}

\begin{document}

{\bf Names} (First only, no student numbers):\\


Before you begin you should have read and worked through Lab 2.

\noindent{\bf All questions should be done by hand (not by computer)
  and show your steps.}

\begin{enumerate}
\item Error terms in the backwards Euler
\begin{enumerate}
\item Derive the error term for the backward difference formula using Taylor series, and hence show that it is first order.
\vspace{3in}

\item How does the constant in front of the leading order error term differ from that for the forward difference formula? Relate this back to the results plotted in the Lab (Figure error), where these two formulae were used to derive difference schemes for the heat conduction problem. (i.e., explain why backward euler is an overestimate and forward euler is an underestimate of the exact solution)
\vspace{3in}
\end{enumerate}
\item For the equation $y = sin(\lambda t)$ derive the following:
\begin{enumerate}
\item $dy/dt$ (= $y'(t)$)
\vspace{1in}
\item Write down an implicit scheme approximation for $y'(t_i)$
\vspace{1.5in}
\item Using the approximation $sin(\alpha - \beta) \approx sin(\alpha) - \beta cos(\alpha)$ (true when $\alpha$ and/or $\beta$ follow a particular condition) show that your implicit scheme is consistent with the analytical solution.  
\vspace{2in}
\item What are the conditions under which the approximation $sin(\alpha - \beta) \approx sin(\alpha) - \beta cos(\alpha)$ holds? (do an internet search if none of your group knows offhand). What do these conditions mean (qualitatively) for the accuracy of the implicit scheme from section 2b? 
\end{enumerate}

\vspace{2in}

\end{enumerate}

\end{document}