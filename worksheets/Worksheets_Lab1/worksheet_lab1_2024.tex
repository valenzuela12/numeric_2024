\documentclass[12pt]{article}

\include{../latex_env}

\rhead{Worksheet One on Lab 1}

\begin{document}

{\bf Names} :\\

\noindent{\bf Answer Questions on the Paper (or on a pdf version on your computer)}

\begin{enumerate}
\item An atmospheric scientist is looking at the impact of stationary Rossby waves generated by the Tibetan Plateau.  As a first step they want to find the natural, stationary modes.  To do this they will solve:

\begin{eqnarray}
\left(\ddt + u \ddx + v \ddy \right)\zeta + \beta v &=& 0 \\
{\rm where} \ \zeta &=& -\beta \eta(x)\  {\rm at}\  y = y_o^\pm
\end{eqnarray}
Here $y = y_o + \eta^+$ and $y = y_o + \eta^-$ are the northern and southern profiles of the mountain.

Note that the independent variables in this problem are $x, y, t$, the dependent variable is $\zeta$, parameters and defined values are $u, v, \eta, \beta, y_o$.

For each question below, give an answer and brief (one or two sentence) explanation:
\begin{enumerate}
\item Is this equation linear or nonlinear?\\[48pt]
\item A partial differential equation or an ordinary differential equation?\\[48pt]
\item An initial value problem? A boundary value problem?\\[48pt]
\item On an open domain or a closed domain?\\[48pt]
\end{enumerate}
\clearpage
\item An oceanographer is considering the growth of phytoplankton over time in a one-dimensional model as a function of the nutrients and light they receive.  To do this they will solve:

\begin{equation}
\frac {dP}{dt}= R \exp(\sigma T) P \left(\frac N {\kappa + N} \right) I \exp[-(k+\alpha P)d]
\end{equation}
where $d$ is depth, $t$ is time -- the independent variables, $P$ is phytoplankton concentration -- the dependent variable, $R$ is the growth rate, $T$ is temperature $N$ is nitrate concentration and $I$ is light (insolation) -- all defined values, $\sigma$, $\kappa$, $k$ and $\alpha$ are parameters.

On December 1 the phytoplankton concentration is assumed to be 1 uM and at 30 m depth it is assumed to be 0.1 uM.

For each question below, give an answer and brief (one or two sentence) explanation:
\begin{enumerate}
\item Is this equation linear or nonlinear?\\[48pt]
\item A partial differential equation or an ordinary differential equation?\\[48pt]
\item An initial value problem? A boundary value problem?\\[48pt]
\item On an open domain or a closed domain?\\[48pt]
\end{enumerate}
\end{enumerate}

\end{document}
